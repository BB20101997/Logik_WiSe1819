\section*{Aufgabe 1}

\subsection*{Eine Formel in NNF umwandeln}

$\varphi = \neg ((X_0\wedge\neg X_1) \vee \neg(X_0 \wedge X_1))$

\subsubsection*{1.}

\begin{tabular}{rcl}
$BNF2NNF(\varphi)$ & $=$ & $BNF2NNF(\neg ((X_0\wedge\neg X_1) \vee \neg(X_0 \wedge X_1)))$ \\
& $=$ & $NegNNF(BNF2NNF((X_0\wedge\neg X_1) \vee \neg(X_0 \wedge X_1)))$ \\
& $=$ & $NegNNF(BNF2NNF((X_0\wedge\neg X_1)) \vee BNF2NNF(\neg(X_0 \wedge X_1)))$ \\
& $=$ & $NegNNF((BNF2NNF(X_0) \wedge BNF2NNF(\neg X_1)) \vee $ \\
&     & $NegNNF(BNF2NNF(X_0 \wedge X_1)))$ \\
& $=$ & $NegNNF((X_0 \wedge NegNFF(BNF2NNF(X_1))) \vee $ \\
&     & $NegNNF(BNF2NNF(X_0) \wedge BNF2NNF(X_1)))$ \\
& $=$ & $NegNNF((X_0 \wedge NegNFF(X_1)) \vee $ \\
&     & $NegNNF(X_0 \wedge X_1))$ \\
& $=$ & $NegNNF((X_0 \wedge \neg(X_1)) \vee $ \\
&     & $(NegNNF(X_0) \vee NegNNF(X_1)))$ \\
& $=$ & $NegNNF((X_0 \wedge \neg(X_1)) \vee $ \\
&     & $(\neg X_0 \vee \neg X_1))$ \\
& $=$ & $NegNNF(X_0 \wedge \neg X_1) \wedge $ \\
&     & $NegNNF(\neg X_0 \vee \neg X_1)$ \\
& $=$ & $(NegNNF(X_0) \vee NegNNF(\neg X_1)) \wedge $ \\
&     & $(NegNNF(\neg X_0) \wedge NegNNF(\neg X_1))$ \\
& $=$ & $(\neg X_0 \vee X_1) \wedge (X_0 \wedge X_1)$ \\
\end{tabular}

\subsubsection*{2.}

\begin{tabular}{rcl}
$BNF2NNF2(\varphi)$ & $=$ & $BNF2NNF2(\neg ((X_0\wedge\neg X_1) \vee \neg(X_0 \wedge X_1)))$ \\
& $=$ & $BNF2NNF2\textit{-}neg((X_0\wedge\neg X_1) \vee \neg(X_0 \wedge X_1))$ \\
& $=$ & $BNF2NNF2\textit{-}neg(X_0\wedge\neg X_1) \wedge BNF2NNF2\textit{-}neg(\neg(X_0 \wedge X_1))$ \\
& $=$ & $(BNF2NNF2\textit{-}neg(X_0)\wedge BNF2NNF2\textit{-}neg(\neg X_1)) \wedge$ \\
& & $ BNF2NNF2(X_0 \wedge X_1)$ \\
& $=$ & $(\neg X_0\wedge  BNF2NNF2(X_1)) \wedge $\\
&     & $(BNF2NNF2(X_0) \wedge BNF2NNF2(X_1))$ \\
& $=$ & $(\neg X_0\wedge  X_1) \wedge (X_0 \wedge X_1)$ \\
\end{tabular}

\pagebreak

\subsection*{Komplexität der Umwandlung in NNF untersuchen}
\par
\texttt{
	LaufBNF2NNF($\varphi$) \\
	Precondition: $\varphi$ in BNF \\
	case $\varphi$ of \\
		$\bot$: return 1 \\
		$\top$: return 1 \\
		$X_i$: return 1 \\
		$\neg \xi$ : return LaufBNF2BNNF($\xi$)+LaufNegNNF(BNF2BNNF($\xi$))+1 \\
		$\Phi_0 \vee \Phi_1$ : return LaufBNF2NNF($\Phi_0$)+ LaufBNF2NNF($\Phi_1$) +1 \\
		$\Phi_0 \wedge \Phi_1$ : return LaufBNF2NNF($\Phi_0$)+ LaufBNF2NNF($\Phi_1$) +1\\
}



\texttt{
	LaufNegNNF($\varphi$)\\
	Precondition: $\varphi$ in NNF \\
	case $\varphi$ of \\
		$\bot$: return 1 \\
		$\top$: return 1 \\
		$X_i$: return 1 \\
		$\neg X_i$: return 1 \\
		$\Phi_0 \vee \Phi_1$ : return LaufBNF2NNF($\Phi_0$)+ LaufBNF2NNF($\Phi_1$) +1 \\
		$\Phi_0 \wedge \Phi_1$ : return LaufBNF2NNF($\Phi_0$)+ LaufBNF2NNF($\Phi_1$) +1\\
}



\texttt{
	LaufBNF2NNF2($\varphi$) \\
	Precondition: $\varphi$ in BNF \\
	case $\varphi$ of \\
		$\bot$: return 1 \\
		$\top$: return 1 \\
		$X_i$: return 1 \\
		$\neg \xi$ : return LaufBNF2BNNF2($\xi$)+1 \\
		$\Phi_0 \vee \Phi_1$ : return LaufBNF2NNF2($\Phi_0$)+ LaufBNF2NNF2($\Phi_1$) +1 \\
		$\Phi_0 \wedge \Phi_1$ : return LaufBNF2NNF2($\Phi_0$)+ LaufBNF2NNF2($\Phi_1$) +1\\
}

Die Laufzeit von BNF2NNF ist Linear zur Anzahl der Knoten die $\varphi$ als Baum hat, wenn $\varphi$ keine Negationen beinhaltet.
Wenn $\varphi$ Negationen beinhaltet kann die Laufzeit von BNFNNF Quadratisch zur der Anzahl der Knoten werden.

Die Laufzeit von BNF2NNF2 ist Linear zur Anzahl der Knoten die $\varphi$ als Baum hat.

Sei $K$ die Zahl der Knoten des Baumes der die aussagenlogische Formel repräsentiert.
Dann gilt:

$O(BNF2NNF) = K^2$ 

$O(BNF2NNF2) = K$ 

