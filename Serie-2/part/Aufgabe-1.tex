\section*{Aufgabe 1}


% $\varphi = (x_j \wedge x_{n+j}) \vee (\bigvee (x_1 \wedge \neg x_{n+1} , x_i \wedge \neg x_{n+i}) \wedge (x_{n+i+1} \wedge \neg x_i+1))$ 
$\varphi = \bigwedge_{i=0}^{n-1}((x_i \rightarrow x_{n+i}) \vee \bigvee_{j=i+1}^{n-1}( x_{n+j} \wedge \neg x_{j}))$ \\



Beide Summen sind als Binärzahlen interpretierbar, da es egal ist was ein Bit in der 2. Zahl ist wenn das ensprechende Bit in der 1. Zahl 0 ist muss nur der Fall betrachtet werden wenn ein Bit der 1. Zahl den Wert 1 hat. Wenn auch das Bit an derselben stelle in der 2. Zahl 1 ist gilt gleichheit in diesem Wert.\\

Dies soll gezeigt werden durch $(x_i \rightarrow x_{n+i})$ .\\


Nun muss nur noch der Fall betrachtet werden wo in der 1. Zahl ein Bit 1 ist selbiges Bit in der Zweiten Zahl aber null, da $2^n-1 = \sum_{i=1}^{n-1}2^i$ gilt, reicht es ein Bit mit höherem Index in der 2. Zahl  existiert (ohne entsprechendes Bit in der ersten Zahl).\\

Dies zeigt $ \bigvee_{j=i+1}^{n-1}( x_{n+j} \wedge \neg x_{j})$. \\



Andere Formel: $ \varphi = \neg \bigvee_{i=0}^{n-1}(x_i \wedge \neg x_{n+i} \wedge \bigwedge_{j=i+1}^{n-1} (x_j \wedge x_{n+j}))$ 
Hier wird nur gesucht ob ein Bit in der ersten Zahl existiert ohne dass ein enstrechendes Bit oder höher, die nicht "belegt" ist, in der zweiten Zahl existiert.