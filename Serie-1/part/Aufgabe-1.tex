\section*{Aufgabe 1}

\subsection*{Präfix Notation}

Sei $A$ die Menge der modalen aussagenlogischen Formeln als Zeichenketten in Präfix Notation.


\subsubsection*{Basismenge}

Dann sind:

$\bot \in A$

$\top \in A$

$\forall i \in \mathbb{N} X_i \in A$

\subsubsection*{Induktionsregel}

Die Modalitäten $\square$ und $\lozenge$ als Jukter der Stelligkeit 1 zu betrachten.

Sei $J$ ein $n$ Stelliger Junktoren und $f_0, ... , f_{n-1} \in A$  , dass

$J(f_0, ... , f_{n-1}) \in A$


\subsection*{Formel als Baum}

Sei $A$ die Menge der modalen aussagenlogischen Formeln in Form eines Baumes.

\subsubsection*{Basismenge}

Die Bäume mit einem Knoten, welche für $i \in \mathbb{N}$ mit $\top, \bot$ oder $X_i$ beschriftet sind, sind in $A$.

\subsubsection*{Induktionsregel}

Die Modalitäten $\square$ und $\lozenge$ als Jukter der Stelligkeit 1 zu betrachten.

Sei $J$ ein $n$ Stelliger Junktor und $f_0, ... , f_{n-1} \in A$, 

dann ist der Baum dessen Wurzel mit $J$ beschriftet ist und dessen Kinder $f_0, ... , f_{n-1}$ sind in $A$.
