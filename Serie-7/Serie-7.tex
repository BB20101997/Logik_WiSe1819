\documentclass[12pt,a4paper,oneside]{article} %% Dokumenten Parameter und Art des Dokuments
\usepackage[utf8x]{inputenc} %% Diese Datei ist im utf8 Format dies ist hier damit Latex uns versteht
\usepackage[ngerman]{babel} %% Rechtschreib prüfung
\usepackage{hyperref}
\usepackage{amsmath} %% Packet zur verwendung Mathematischer Formeln
\usepackage{amsfonts}
\usepackage{amssymb} %% Packet zur verwendung Mathematischer Symbole
\usepackage{mathtools}
\usepackage{microtype} %% Sorgt für besseren umgang mit zu lange/kurzen Zeilen
\usepackage{pdfpages} %% Zum einfügen eines PDf dokuments

\usepackage{listings}
\lstset{
  basicstyle=\ttfamily,
  mathescape
}

%\usepackage{mathabx} % for the =| symbol
% Setup the matha font (from mathabx.sty)
\DeclareFontFamily{U}{matha}{\hyphenchar\font45}
\DeclareFontShape{U}{matha}{m}{n}{
      <5> <6> <7> <8> <9> <10> gen * matha
      <10.95> matha10 <12> <14.4> <17.28> <20.74> <24.88> matha12
      }{}
\DeclareSymbolFont{matha}{U}{matha}{m}{n}
% Define the vDash and Dashv characters from that font (from mathabx.dcl)
% This can ovveride symbols by giving it an existing name
% \DeclareMathSymbol{\name}{type}{font}{slot}
\DeclareMathSymbol{\vDash}{3}{matha}{"28}
\DeclareMathSymbol{\Dashv}{3}{matha}{"29}

\title{Logik Serie 7}
\author{Bennet Bleßmann, Sven Korfmann}

\begin{document}
\maketitle

\setlength{\parindent}{0pt}	
\section*{Aufgabe 1}


% $\varphi = (x_j \wedge x_{n+j}) \vee (\bigvee (x_1 \wedge \neg x_{n+1} , x_i \wedge \neg x_{n+i}) \wedge (x_{n+i+1} \wedge \neg x_i+1))$ 
$\varphi = \bigwedge_{i=0}^{n-1}((x_i \leftrightarrow x_{n+i}) \vee \bigvee_{j=i+1}^{n-1}( x_i \wedge x_{n+j} \wedge \neg x_{j}))$ 



Da für Variablen die genau n Werte entfernt sind gilt, dass sie beide Summen, wenn sie 1 sind, um genau den selben Wert erhöhen gilt bei ihnen die Gleichheit der kleiner gleich Relation. Dies zeigt $(x_i \leftrightarrow x_{n+i})$ .\\


Nun müssen nur noch die Variablen betrachtet werden die nicht in beiden Summen dieselbe Potenz von 2 darstellen, da $2^n-1 = \sum_{i=1}^{n-1}2^i$ gilt, reicht es dass zu jeder Variable in der ersten Summe (ohne entsprechende Variable in der zweiten Summe)  eine Variable mit höherem Index in der 2. Summe  existiert (ohne entsprechende Variable in der ersten Summe).
Dies zeigt $ \bigvee_{j=i+1}^{n-1}( x_i \wedge x_{n+j} \wedge \neg x_{j})$. \\



Andere Formel: $ \varphi = \neg \bigvee_{i=0}^{n-1}(x_i \wedge \neg x_{n+i} \wedge \bigwedge_{j=i+1}^{n-1} (x_i \wedge x_{n+i}))$ 
Hier wird nur gesucht ob eine Potenz von 2 in der ersten Summe existiert ohne dass eine enstrechende Potenz oder höhere, die nicht "belegt" ist, in der zweiten Summe existiert.	
\section*{Aufgabe 2}

\subsection*{Variablen in Termen Finden)}

\subsubsection*{Inductionsanfang}
$vars(x_i)=\{ x_i\} $ für alle $i \in \mathbb{N}$


\subsubsection*{Induktionschritt}
$vars(f(t_0, \ldots , t_{n-1}))= \cup_{i=0}^{n} vars(t_i)$

\subsection*{Quantorenrang definieren}
\subsubsection*{Inductionsanfang}
$qrank(x_i)=0$\\
$qrank(f(t_0, \ldots ,t_{n-1} ))=0$\\
$qrank(R(t_0, \ldots , t_{n-1})=0$\\
\subsubsection*{Induktionschritt}
$qrank(C(\varphi_{0}, \ldots , \varphi_{n-1}))= max\{ qrank(\varphi_{0}) , \ldots , qrank(varphi_{n-1}) \}$\\
$qrank(Qx_i (\varphi))= 1+ qrank(\varphi)$

	
\section*{Aufgabe 3}

\subsection*{1)}

Sei $\varphi$ eine aussagenlogische Formel und $V = vars(\varphi),X_i \in V, W = V \setminus \{X_i\}$.

Sein weiter $\varphi_0 = \varphi\{X_i \mapsto \bot\}$ 
und $\varphi_1 = \varphi\{X_i \mapsto \top\}$ 

\subsubsection*{Fall: $\varphi$ erfüllbar}
Sei $\varphi$ erfüllbar mit einer 1-Belegung $\beta$.
Dann 

\subsubsection*{Fall: $\varphi$ nicht erfüllbar}

\subsection*{2)}	
\end{document}