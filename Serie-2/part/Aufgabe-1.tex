\section*{Aufgabe 1}


% $\varphi = (x_j \wedge x_{n+j}) \vee (\bigvee (x_1 \wedge \neg x_{n+1} , x_i \wedge \neg x_{n+i}) \wedge (x_{n+i+1} \wedge \neg x_i+1))$ 
$\varphi = \bigwedge_{i=0}^{n-1}((x_i \wedge x_{n+i}) \vee \bigvee_{j=i+1}^{n-1}( x_i \wedge x_{n+j} \wedge \neg x_{j}))$ 



Da für Variablen die genau n werte entfernt sind gilt, dass sie beide Summen, wenn sie 1 sind, um genau den selben Wert erhöhen gilt bei ihnen die Gleichheit der kleiner gleich Relation. Dies zeigt $(x_i \wedge x_{n+i})$ .\\


Nun müssen nur noch die Variablen betrachtet werden die nicht in beiden Summen dieselbe Potenz von 2 darstellen, da $2^n-1 = \sum_{i=1}^{n-1}$ gilt, reicht es dass zu jeder Variable (ohne n entfernte Variable) in der ersten Summe eine Variable mit höherem index (ohne n entfernte Variable) in der 2. Summe  existiert
Dies zeigt $ \bigvee_{j=i+1}^{n-1}( x_i \wedge x_{n+j} \wedge \neg x_{j})$. \\



Andere Formel: $ \varphi = \neg \bigvee_{i=0}^{n-1}(x_i \wedge \neg x_{n+i} \wedge \bigwedge_{j=i+1}^{n-1} (x_i \wedge x_{n+i}))$ 
Hier wird nur gesucht ob eine Potenz von 2 in der ersten Summe existiert ohne dass eine enstrechende Potenz oder höhere, die nicht "belegt" ist, in der zweiten Summe existiert.