\section*{Aufgabe 3}

\subsection*{1.}

\subsubsection*{Beweis für Gleichheit $(t_0 \dot= t_1)$}

Seien $t_0$ und $t_1$ Terme, weiter sei $\varphi = t_0 \dot= t_1$.

Sein $\beta_0$ und $\beta_1$ $\mathcal{A}$-Belegungen mit $\beta_0|_{fvars(\varphi)}$ = $\beta_1|_{fravs(\varphi)}$.

\hfill

Wir schließen

$\beta_0|_{fvars(t_i)} = \beta_1|_{fvars(t_i)},$ für $i \in \{0,1\}$ 

da $fvars(t_i) \subseteq fvars(\varphi)$ nach der Definition von fvars.

\hfill

Nach dem ersten Lemma der Section "Proof preperation" der Wiki Seite " Coincidence lemma in first-order logik",
schließen wir

$[\![ t_i ]\!]_{\beta_0}^\mathcal{A} = [\![ t_i ]\!]_{\beta_1}^\mathcal{A}$ für alle $i \in \{0,1\}$ (*)

\hfill

Wir können nun Folgern

$\mathcal{A},\beta_0 \vDash \varphi$ 

genau dann wenn $[\![ t_0 ]\!]_{\beta_0}^\mathcal{A} = [\![ t_1 ]\!]_{\beta_0}^\mathcal{A}$  wegen der definition von $\varphi$ und der semantik

genau dann wenn $[\![ t_0 ]\!]_{\beta_1}^\mathcal{A} = [\![ t_1 ]\!]_{\beta_1}^\mathcal{A}$ (*)

genau dann wenn $\mathcal{A},\beta_1 \vDash \varphi$ wegen der definition von $\varphi$ und der semantik

\subsubsection*{Beweis für Junktoren $(C(\varphi_0,...,\varphi_{n-1}))$ }

Sei $t_i$ ein Term für alle $i \in \{0,...,n-1\}$ und $C$ ein n-Stelliger Junktor.

Sei $\varphi = C(t_0,...,t_{n-1})$.

Sein $\beta_0$ und $\beta_1$ $\mathcal{A}$-Belegungen mit $\beta_0|_{fvars(\varphi)} = \beta_1|_{fravs(\varphi)}$.

\hfill

Wir schließen

 $\beta_0|_{fvars(t_i)} = \beta_1|_{fvars(t_i)}$ für alle $i \in \{0,...,n-1\}$ 
 
 da $fvars(t_i) \subseteq fvars(\varphi)$ nach der Definition von fvars.
 
\hfill

Nach dem ersten Lemma der Section "Proof preperation" der Wiki Seite " Coincidence lemma in first-order logik",
schließen wir 

$[\![ t_i ]\!]_{\beta_0}^\mathcal{A} = [\![ t_i ]\!]_{\beta_1}^\mathcal{A}$ für $i \in \{0,n-1\}$(*)

\hfill

Wir können nun Folgern

$[\![\varphi]\!]_{\beta_0}^\mathcal{A}$ 

=  $C^{\mathcal{A}}([\![ t_0 ]\!]_{\beta_0}^\mathcal{A},...,[\![ t_{n-1} ]\!]_{\beta_0}^\mathcal{A})$ wegen der definition von $\varphi$ und der semantik

=  $C^{\mathcal{A}}([\![ t_0 ]\!]_{\beta_1}^\mathcal{A},...,[\![ t_{n-1} ]\!]_{\beta_1}^\mathcal{A})$ (*)

= $[\![\varphi]\!]_{\beta_1}^\mathcal{A}$ wegen der definition von $\varphi$ und der semantik


%zz \forall \beta_0 \beta_1 with [| \beta_0|]|_fvars(\varphi)


\subsection*{2.}

\subsubsection*{Definition für $fvars$ auf Formel Mengen}

Sei $\Phi$ eine Menge von Formeln.

Dann sei $fvars(\Phi) = \bigcup\limits_{\varphi \in \Phi} fvars(\varphi)$.

\subsection*{Koinzidenzlemma für Mengen von Formeln}

Sei $\Phi$ eine Menge von Formeln.

Sein $\beta_0$ und $\beta_1$ $\mathcal{A}$-Belegungen mit $\beta_0|_{fvars(\Phi)} = \beta_1|_{fvars(\Phi)}$.

\hfill

Nach der definition von fvars auf Formel Mengen gilt

für alle $\varphi \in \Phi: \beta_0|_{fvars(\varphi)} = \beta_1|_{fvars(\varphi)}$,

da $fvars(\varphi) \subseteq fvars(\Phi)$.

\hfill

Nach dem ersten Lemma der Section "Proof preperation" der Wiki Seite " Coincidence lemma in first-order logik",
schließen wir 

$[\![ \varphi ]\!]_{\beta_0}^\mathcal{A} = [\![ \varphi ]\!]_{\beta_1}^\mathcal{A}$ für alle $\varphi \in \Phi$(*)

Wir können nun Folgern

$\mathcal{A},\beta_0 \vDash \Phi$

genau dann wenn $[\![ \varphi ]\!]_{\beta_0}^\mathcal{A} = 1$ für alle $\varphi \in \Phi$ wegen der Semantik

genau dann wenn $[\![ \varphi ]\!]_{\beta_1}^\mathcal{A} = 1$ für alle $\varphi \in \Phi$ (*)

genau dann wenn $\mathcal{A},\beta_1 \vDash \Phi$ wegen der Semantik

