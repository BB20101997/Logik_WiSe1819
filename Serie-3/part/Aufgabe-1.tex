\section*{Aufgabe 1}

\subsection*{comp 1)}

Sei $\beta$ beliebige Variablenbelegung dann gilt:\\

$[ [ X_0 \wedge \neg X_0 ] ]_\beta = and([[X_0 ]]_\beta , [[\neg X_0]]_\beta)$ Da Semantik von $\wedge$.\\
$=and([[X_0 ]]_\beta , neg([[ X_0]]_\beta))$ Da Semantik von $\neg$ .\\
$=0$ Da min 1 Wert 0. \\
$=[[\bot]]_\beta$ Semantik von $\bot$.\\
$\Rightarrow  X_0 \wedge \neg X_0 \Dashv \vDash \bot $

\subsection*{comp 2)}
Sei $\beta$ beliebige Variablenbelegung dann gilt:\\

$[ [ X_0 \vee \neg X_0 ] ]_\beta = or([[X_0 ]]_\beta , [[\neg X_0]]_\beta)$ Da Semantik von $\vee$.\\
$=or([[X_0 ]]_\beta , neg([[ X_0]]_\beta))$ Da Semantik von $\neg$ .\\
$=1$ Da min 1 Wert 1. \\
$=[[\top]]_\beta$ Semantik von $\top$.\\
$\Rightarrow  X_0 \vee \neg X_0 \Dashv \vDash \top $


\subsection*{rev)}
Eigenschaft, die die booleschen Funktionen and und And miteinander in Beziehung setzt:\\
$and(And([[X_0]]_\beta, \ldots , [[X_n]]_\beta), [[X_{n+1}]]_\beta)=And([[X_0]]_\beta , \ldots , [[X_{n+1}]]_\beta)$\\
für alle Variablenbelegungen $\beta$

\subsubsection*{Induktionsanfang}
Sei $\varphi = \bigwedge ()$ Dann gilt:\\
$\bigwedge () = 1 $ Nach Vorlesung.\\
$= \top$ Semantik von $\top$

\subsubsection*{Induktionsvorraussetzung}
Sei $n \in N $ mit $\bigwedge (X_0, \ldots , X_{n-1}) \Dashv \vDash \varphi $\\
Mit $\varphi$ der form $\varphi=( \ldots (X_0 \wedge X_1) \wedge X_2 ) \ldots \wedge X_{n-1})$

\subsubsection*{Induktionsschritt} 
Sei $\bigwedge (X_0, \ldots , X_{n})$, sei $\beta$ Varablenbelegung dann gilt:\\
$[[\bigwedge (X_0, \ldots , X_{n})]]_\beta = And([[X_0]]_\beta, \ldots , [[X_n]]_\beta)$ Semantik von $\bigwedge$\\
$= and(And([[X_0]]_\beta, \ldots , [[X_{n-1}]]_\beta ), [[X_n]]_\beta)$ Verbindung Zwischen $And$ und $and$. \\
$= and([[\varphi]]_\beta, [[X_n]]_beta)$ Nach Induktionsvorraussetzung \\
$[[\varphi \wedge X_n]]_\beta$\\
$\Rightarrow \bigwedge (X_0, \ldots , X_{n}) \Dashv \vDash \varphi \wedge X_n $