\documentclass[12pt,a4paper,oneside]{article} %% Dokumenten Parameter und Art des Dokuments
\usepackage[utf8x]{inputenc} %% Diese Datei ist im utf8 Format dies ist hier damit Latex uns versteht
\usepackage[ngerman]{babel} %% Rechtschreib prüfung
\usepackage{hyperref}
\usepackage{amsmath} %% Packet zur verwendung Mathematischer Formeln
\usepackage{amsfonts}
\usepackage{amssymb} %% Packet zur verwendung Mathematischer Symbole
\usepackage{mathtools}
\usepackage{microtype} %% Sorgt für besseren umgang mit zu lange/kurzen Zeilen
\usepackage{pdfpages} %% Zum einfügen eines PDf dokuments

\usepackage{listings}
\lstset{
  basicstyle=\ttfamily,
  mathescape
}

%\usepackage{mathabx} % for the =| symbol
% Setup the matha font (from mathabx.sty)
\DeclareFontFamily{U}{matha}{\hyphenchar\font45}
\DeclareFontShape{U}{matha}{m}{n}{
      <5> <6> <7> <8> <9> <10> gen * matha
      <10.95> matha10 <12> <14.4> <17.28> <20.74> <24.88> matha12
      }{}
\DeclareSymbolFont{matha}{U}{matha}{m}{n}
% Define the vDash and Dashv characters from that font (from mathabx.dcl)
% This can ovveride symbols by giving it an existing name
% \DeclareMathSymbol{\name}{type}{font}{slot}
\DeclareMathSymbol{\vDash}{3}{matha}{"28}
\DeclareMathSymbol{\Dashv}{3}{matha}{"29}

\title{Logik Serie 8}
\author{Bennet Bleßmann, Sven Korfmann}

\begin{document}
\maketitle

\setlength{\parindent}{0pt}
\section*{Aufgabe 1}

\subsection*{1)}
$X_0 \rightarrow X_1 , X_0 \rightarrow \neg X_1 \vdash \neg X_0$\\

\begin{tabular}{c |c | c}
1 &$X_0 \rightarrow X_1 , X_0 \rightarrow \neg X_1$ & Premise\\
\hline
1.1 & $X_0$ & Assumption\\
1.2 & $X_0 \rightarrow X_1$ & premise \\
1.3 & $X_1$ & $(\rightarrow e)$ With  and 1.1 and 1.2 \\

1.4 & $X_0 \rightarrow \neg X_1$ & premise \\
1.5 & $\neg X_1$ & $(\rightarrow e)$ With  and 1.1 and 1.4 \\
1.6 & $\bot$ & $(\bot i)$ with 1.3 and 1.5\\
\hline
2 & $\neg X_0$ & $(\neg i)$ with 1

\end{tabular}


\subsection*{2)}
$\neg X_0 \vee X_1 \dashv \vdash X_0 \rightarrow X_1 $\\

\begin{tabular}{c |c | c}
1 & $X_0 \rightarrow X_1 $ & Premise\\
\hline
1.1 &  $X_0$ & Assumption\\
1.2 & $X_0 \rightarrow X_1 $ & Premise\\
1.3 & $X_1$ & $(\rightarrow e)$ With 1.1. and 1.2\\
\hline
2.1 & $\neg X_0$ & Assumption\\
3 & $\neg X_0 \vee X_1$ & $(\vee i)$ with 1 and 2\\


\end{tabular}\\


\begin{tabular}{c |c | c}
1 & $\neg X_0 \vee X_1$ & Premise\\
\hline
1.1 &  $X_0$ & Assumption\\
\hline
1.1.1 & $ X_1$ & Assumption\\
2 & $ X_0 \rightarrow X_1 $ & $(\rightarrow i)$ With 1 \\
\end{tabular}

\subsection*{3)}
$X_0 \vee X_1 , \neg X_1 \vee X_2 \vdash X_0 \vee X_2$\\

\begin{tabular}{c |c | c}
1 & $X_0 \vee X_1 , \neg X_1 \vee X_2$ & Premise\\
\hline
1.1 & $\neg X_0 $ & Assumption \\
1.1.1 & $\neg X_2 $ & Assumption \\
1.1.2 & $\bot$
\end{tabular}
	
\section*{Aufgabe 2}

$[zz. \neg((X_0 \rightarrow X_2) \rightarrow X_3) \wedge X_4 \Dashv \vDash (X_4 \wedge \neg X_3) \wedge (X_0 \rightarrow X_2)]$

$\neg((X_0 \rightarrow X_2) \rightarrow X_3) \wedge X_4$

\begin{tabular}{llc}
$\Dashv \vDash$ & 
$\neg(\neg(X_0 \rightarrow X_2) \vee X_3) \wedge X_4$ 
  & congruence lemma with $(rew \rightarrow)$ applied to \\
& & $\neg(X_5) \wedge X_4$ with \\
& & $\{X_5 \mapsto (X_0 \rightarrow X_2) \rightarrow X_3\}$ \\ 
& & $\{X_5 \mapsto \neg (X_0 \rightarrow X_2) \vee X_3\}$   \\
%
$\Dashv \vDash$ &
$(\neg\neg( X_0 \rightarrow X_2) \wedge \neg X_3) \wedge  X_4$ 
  & congruence lemma with $(demo)$ applied to \\
& & $X_5 \wedge  X_4$ with \\
& & $\{X_5 \mapsto \neg(\neg(X_0 \rightarrow X_2) \vee X_3)\}$ \\ 
& & $\{X_5 \mapsto (\neg \neg (X_0 \rightarrow X_2) \wedge \neg X_3)\}$ \\
%
$\Dashv \vDash$ &
$(( X_0 \rightarrow X_2) \wedge \neg X_3) \wedge  X_4$ 
  & congruence lemma with $(done)$ applied to \\
& & $(X_5 \wedge \neg X_3) \wedge  X_4$ with \\
& & $\{X_5 \mapsto \neg\neg(X_0 \rightarrow X_2)\}$ \\ 
& & $\{X_5 \mapsto (X_0 \rightarrow X_2)\}$ \\
%
$\Dashv \vDash$ & $(X_0 \rightarrow X_2) \wedge (\neg X_3 \wedge  X_4)$  & $(ass)$\\
%
$\Dashv \vDash$ & $(\neg X_3 \wedge  X_4) \wedge (X_0 \rightarrow X_2)$  & $(comm)$\\
% 
$\Dashv \vDash$ &
$(X_4 \wedge \neg X_3) \wedge (X_0 \rightarrow X_2)$ 
  & congruence lemma with $(comm)$ applied to \\
& & $(X_5) \wedge (X_0 \rightarrow X_2)$ with \\
& & $\{X_5 \mapsto \neg X_3 \wedge X_4)\}$ \\ 
& & $\{X_5 \mapsto X_4 \wedge \neg X_3\}$ \\
\end{tabular}
\section*{Aufgabe 3}

\subsection*{1)}

Sein $\varphi$ und $\psi$ $\wedge$-Formeln mit $\varphi \Dashv \vDash \psi$.

Angenommen $var(\varphi) \neq var(\psi).$

Dann existiert $X_i \in (var(\varphi) \cup var(\psi)) \setminus (var(\varphi) \cap var(\psi))$.

Sei $\beta$ 1-Belegung für $\varphi$ und somit per Vorraussetzung auch 1-Belegung für $\psi$.
Sei $\beta'(X_j) = \begin{cases}
	\beta(X_j) &\textit{, für} j \neq i \\
	0		  &\textit{, für} j = i
	\end{cases}
$ 

dann gilt $[\![\varphi]\!]_{\beta'} \neq [\![\psi]\!]_{\beta'}$

da für die Formel die $X_i$ nicht enthält $\beta'$ eine 1-Belegung ist und für die Formel die $X_i$ enthält $\beta'$ keine 1-Belegung ist. 


%\subsubsection*{Induktionsanfang}
%Seien $\varphi , \psi$ $\wedge$-Formeln mit 1 Variablen dann gilt für ein $i \in N $:\\
%$\varphi \Dashv \vDash X_i \Dashv \vDash \psi$ damit gilt $vars(\varphi)=\{ X_i \}=vars(\psi)$\\

%\subsubsection*{Vorraussetzung}

\subsection*{2)}

$\top$ trägt nicht zum Wahrheitswert einer $\wedge$-Formel,
 auch ändert $\top$ nicht der Variablemenge einer $\wedge$-Formel,
 also kann $\top$ als Basiselement zugelassen werden ohne das 1. geändert werden muss.

$\bot$ sorgt in einer $\wedge$-Formel dass diese nicht erfüllbar wird,
dies bedeutet das für jede $\wedge$-Formel $\varphi$ welche $\bot$ und eine Variable $X_i$ enthält,
 $\varphi \Dashv \vDash \bot$  gilt, aber auch $vars(\varphi) \supseteq \{X_i\} \neq \{\} = vars(\bot)$ und somit 1. wiederspricht.
 
Für die Aufnahme von $\bot$ sollte 1. so angepasst werden, 
	dass für $\varphi$ und $\psi$ welche kein $\bot$ enthalten die ursprüngliche Formel gilt.
	
\end{document}